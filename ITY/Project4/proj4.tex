\documentclass[a4paper, 11pt]{article}
\usepackage[text={17cm,24cm}, left=2cm, top=3cm]{geometry}
\usepackage[czech,shorthands=off]{babel}
\usepackage[utf8]{inputenc}
\usepackage{times}
\usepackage[hidelinks]{hyperref}
\usepackage{cite}
\usepackage{csquotes}

\begin{document}

%Title page
\begin{titlepage}
    \begin{center}
    \Huge\textsc{Vysoké učení technické v~Brně\\}
        \huge \textsc{Fakulta informačních technologií}\\
        \vspace{\stretch{0.382}}
        \LARGE{Typografie a~publikování\,--\,4. projekt}\\
        \Huge{Bibliografické citace}
        \vspace{\stretch{0.618}}
        
        {\Large 12. dubna 2024 \hfill Martin Kováčik}
        
    \end{center}
\end{titlepage}

\section{Typografie}

\subsection{Definice}
\enquote{\texttt{Typografie} je disciplína zabývající~se písmem, především jeho správným výběrem, použitím a~sazbou. Cílem typografie je zajistit čtenáři \emph{snazší čtení}, efektivnější \emph{vnímání čteného textu} a~případně i~vyloučit možné chyby a~nejednoznačnosti plynoucí z~více možných zápisů téže věty.}\cite{Strafelda}

\subsection{Historie}
Slovo \enquote{typografie} má původ v~řeckých slovech \enquote{typos} (znamenající písmo) a~\enquote{graphe} (což znamená psaní). Existuje několik teorií o~původu typografie, ale~není pochyb o~tom, že umění tisku pomocí oddělených typů začalo v~německé Mohuči kolem let~1450 a~odtud se rychle rozšířilo po celém světě.\cite{GressEdmundG1917}

\subsubsection{Modernizace typografie}
První kroky vedoucí směrem k~digitální tedy \texttt{počítačové sazbě} se uskutečnily až v~50. letech~20. století. Teprve \texttt{Donald E. Knuth} představil sázecí systém \TeX~v~roce~1978 a~tím přinesl kvalitní digitální sazbu.\cite{inproceedings}

\subsection{Typografie webdesignu}
Správná typografie je zásadním prvkem \texttt{designu webu}, ovlivňuje jeho vzhled i~celkový dojem. Kvalitní práce s~písmem zvyšuje čitelnost a~estetiku stránek. V digitálním prostředí nabývá typografie ještě většího významu, kdy designéři využívají \texttt{moderní trendy a~technologie} pro tvorbu atraktivních typografických prvků.\cite{FinkoTypografieWebdesign}

\subsection{Dnešní písmo}
Původně se typografie zaměřovala na tiskové písmo, ale \emph{v~dnešní digitální éře} nachází nové využití. S~rozvojem internetu a~stále se rozšiřující nabídkou fontů a~jejich úprav se stává běžné, že každý může vytvořit \emph{svůj vlastní font} a sdílet ho online.\cite{Nova2018}

Je vhodné udržovat \texttt{jednotný vzhled} dokumentu tím, že používáte pouze jedno písmo. Pro zvýraznění nebo~odlišení určitých částí můžete využít \emph{kurzívu}, různé tloušťky písma a~podobně. Pro běžný text se obvykle doporučuje velikost písma v rozmezí~\emph{10 až~12} bodů.\cite{FriesenPat2010TPoT}


\subsection{Pravidla typografie}
Pravidla se začala formovat od samého počátku a~postupně se vyvíjela až do současnosti. Jejich \texttt{hlavním cílem vždy bylo usnadnit čtení textu} pro čtenáře. Dodržování pravidel pravopisu je rovněž klíčové spolu s~typografickými normami.\cite{StudijniOporaTypografie2012}

\subsection{Moderní sázení}
Systém \TeX~je \texttt{Turingův úplný programovací jazyk}. To ve zkratce znamená, že pokud uživatelům chybí nějaká funkce, tak si ji mohou sami doprogramovat.\cite{SyropoulosApostolos2004TXaD}

\subsection{Média a formáty}
Tradiční typografie byla spojena \texttt{s~papírem jako primárním médiem}, což poskytovalo pevně daný formát pro tisk. Nicméně v digitální éře moderních zařízení, jako jsou \emph{telefony a~tablety}, se papír stále více vytrácí. Tyto technologie mění způsob, jakým lidé přistupují k~textu a~typografii.\cite{Jirasek2016thesis}

\subsection{Dopad typografie}
Je běžné, že \emph{špatná typografie} může zkazit i~nejkvalitnější texty, zatímco správná typografie podporuje pozornost čtenáře a~\texttt{zlepšuje celkové vnímání obsahu}. V~současném světě plném webových stránek, plakátů a~dalších vizuálních prvků se často setkáváme s různými problémy spojenými s~typografií.\cite{RichTim2003Wtbt}

\newpage
	\bibliographystyle{czechiso}
	\renewcommand{\refname}{Použité zdroje:}
	\bibliography{proj4}

\end{document}